\section{Introduction}
\label{sec.introduction}
This document describes the design and considerations of the \emph{Fluid Photo
Browser} project, representing the second mandatory assignment. The goal of the
project is to design, implement and document a multi-device photo browser. This
will allow users to move photos from smart phones holding personal
information, to a shared/public tabletop where many people can browse the photos
all together. Please find out more about the requirements and
specifications of the project
\href{https://blog.itu.dk/SPVC-E2010/exercises/assignment-2/}{here}\footnote{\url{https://blog.itu.dk/SPVC-E2010/exercises/assignment-2/}}.
You can browse the source code on the project's
\href{http://subversion.assembla.com/svn/fluid_photo_browser/trunk}{svn}\footnote{\url{http://subversion.assembla.com/svn/fluid_photo_browser/}}.

\subsection{Overview}
\label{sec.introduction.overview}
The system consists of an Android device and a tabletop that communicate in
order to exchange images. The main characteristics of the system are: two-way
device communication and remote control of images. The communication protocol is
bidirectional allowing the android phone to send images to the tabletop and vice
versa. Furthermore, the system provides the option to control remotely the
images on the tabletop using the phone. The design and implementation of these
requirements are explained in the following paragraphs.

\subsection{Motivation}
\label{sec.introduction.motivation}
The diversity of devices present nowadays on the market has arisen the
need of developing multi-device applications, that allow owners of different
types of devices to inter-connect with others in diffent ways. We developed
an application which connects 2 devices: an Android phone and a tabletop. This
is an interesting project which gives us a better understanding of Android and
multitouch surface specific applications, but also the challenges encountered
while interconnecting them.
