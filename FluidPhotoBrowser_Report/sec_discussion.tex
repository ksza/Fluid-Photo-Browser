\section{Discussion}
\label{sec.discussion}
While solving the assignment we had some discussions regarding the
\emph{communication mechanism} and the proposed solution for the
\emph{multi-device handling} in the extra step of the assignment.\\ \\
The current implementation of the communication mechanism is rigid. The client
application on the Android phone has to know the exact IP address and port of the
tabletop in order to be able to establish communication. This solution requires
a wireless infrastructure and a way to let the customer
know the IP address of the tabletop.\\ \\
Besides this, the proposed solution for multi-device handling suggests attaching
RFID tags to each phone which should be read by the tabletop and based on this
look up the fixed IP address of the phone in the network. This solution requires
a lot of extra resources and a lot of configuration.\\ \\
We propose a solution based on the BlueTooth (BT) technology. Every Android
phone has an incorporated BT module. We can attach one bluetooth module to the
tabletop and write our server based on that: the server on the tabletop
implements and publishes a \emph{photoBrowsing} service via the BT module. When
the application is started on the Android phone, it will bring up dialog with
the present devices. The user should see the tabletop in the list of devices and
select it to start the communication. A pairing phase might be required before
the communication is possible.\\ \\
In order to implement the above described solution, on the Android side we would
use the existing API for BT and on the server side (MT4J) we propose to use the
BlueCove Java library for Bluetooth\footnote{\url{http://bluecove.org/}}, a
JSR-82 implementation.

